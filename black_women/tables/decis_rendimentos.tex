\begin{table}[htb!]
    \centering
    \caption{Distribuição de rendimentos por sexo e cor - Brasil 2022}
    \begin{tabular}{lcccc}
    \hline
    Decil & \multicolumn{2}{c}{Brancos} & \multicolumn{2}{c}{Negros} \\ \cline{2-5} 
          & Homens      & Mulheres      & Homens      & Mulheres     \\ \hline
    10\%  & 1212        & 916           & 800         & 550          \\
    50\%  & 2000        & 1700          & 1500        & 1250         \\
    90\%  & 7500        & 5000          & 4000        & 3000         \\
    99\%  & 25000       & 17500         & 12000       & 9000         \\ \hline
    \end{tabular}
    \label{tab:decis}
    \begin{floatnotes}
      \item[Fonte:] Pesquisa Nacional por Amostra de Domicílios Contínua. 2º trimestre/2022.
      \item[Notas:] Os valores representam o montante recebido pelo indíviduo que divide a distribuição no ponto indicado. Por exemplo, a terceira linha da primeira coluna indica que 90\% dos homens brancos recebem até R\$ 7.500,00 por mês o que, reciprocamente, significa que esse é o valor mínimo dos 10\% de maior renda neste grupo. \\
      Para o cálculo, foram considerados apenas rendimento do trabalho de pessoas ocupadas com idade entre 14 e 65 anos.
  \end{floatnotes}
    \end{table}